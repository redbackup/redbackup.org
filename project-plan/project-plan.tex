\documentclass[a4paper]{article}
\usepackage[utf8]{inputenc}
\usepackage[a4paper]{geometry}
\geometry{verbose, marginparwidth=15mm, marginparsep=3mm, tmargin=25mm}
\usepackage{tabulary}
\usepackage{enumitem}

\usepackage{url}
\title{Redbackup: Projectplan}
\author{
		Fabian Hauser \\
		Raphael Zimmermann
}
\date{\today}


\begin{document}
\maketitle

\section{Project Overview}
The goal of the Study Project is to provide a theoretical description of an append-only, distributed peer-to-peer data storage as well as a working prototype as described in the problem statement \cite{problemstatement}.

The project setting, vision, goals, as well as all other project boundaries, are described in depth in the problem statement \cite{problemstatement}.

\section{Project Organization}


\section{Project Workflow}


\section{Risk Management}

\section{Infrastructure}


\subsection{Project Management and Development}

For project management, document/code storage and continuous integration/deployment we utilise the corresponding products by Atlassian (JIRA, BitBucket, Bamboo, Crowd)\cite{atlassian-opensource}.

These applications are hosted on our HSR project server \textit{sinv-56017.edu.hsr.ch}, which runs a standard Ubuntu Linux 17.04.

The formal project documents are written in LaTeX and versioned with git. Other documentation (e.g. Meeting Minutes) are written in Markdown and published on the project website.

\subsubsection{Development Tools}

\begin{itemize}
	\item Git $\geq 2.0$, \url{https://git-scm.com/}
	\item to be defined. %TODO: Add project tools when the programming language is defined
\end{itemize}

\subsection{Backup and Data Safety}

An incremental backup of the project server \textit{sinv-56017.edu.hsr.ch} including the source code and documentation is created on an independent system (\textit{pin1262031.hsr.ch}) every night.

As our documents and code is stored in a git repository, they are also distributed on all development systems.



\section{Quality Measures}

\bibliographystyle{abbrv}
\bibliography{refs}

\end{document}