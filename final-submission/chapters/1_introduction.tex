% !TeX spellcheck = en_GB
\chapter{Introduction}
\label{sec:introduction}

\section{Motivation}
% Rough introduction of the topic. Details are covered in Chapter "Background"
In this section, we legitimate this work and explain the value and applicability of our proposed solution.

\subsection{Present situation}
% What (free) backup solutions exist and why do we need another one?
% Has someone else published the same idea before?
With the ongoing fast digitisation, the demand for reliable and easy-to-use backup solutions is growing fast. Not only enterprises, but also individuals take a great interest in securing their digital artefacts.

Today, most individuals and small enterprises have limited choice with regards to data backup storage.

\paragraph{Cloud backup storage} offers many users high quantities of comfortable, easy to use backup storage. In most cases, these solutions post either a encrypted or unencrypted copies of data into a public cloud environment using specialised software.

Examples for such cloud backup storage providers are Dropbox\footnote{\url{https://www.dropbox.com/}} and Crashplan\footnote{\url{https://www.crashplan.com/}}.

\paragraph{Local backup solutions,} for instance external hard disk drives or network attached storage systems (NAS) offer a high level of data privacy. 

Existing software solutions, for instance Borg Backup~\cite{borg-backup}, rdedup~\cite{rdedup} or custom implementations with rsync~\cite{rsync} and ceph~\cite{ceph} allow for safe, deduplicated backups.

An existing solution which is easy to use for the Apple Mac platform, is Apple Time Machine combined with a Apple Time Capsule\footnote{\url{https://www.apple.com/airport-time-capsule/}}.


\subsection{Problem}
% What problem do we solve with our proposal? / Why do we write this concept? New idea?
% Which problems have we found / are common for this topic?
The backup solutions described in the previous section come with several repercussions.

\paragraph{Cloud backup storage} requires a high level of trust in a third party provider. It is not evident, what level of data security and availability a user is provided with.

Furthermore, such solutions may raise privacy concerns or even legal issues, as data is not kept safely on premises but in a data centre, possibly in another legal domain.

\paragraph{Local backup solutions} require considerable administrative effort, e.g. by managing and exchanging hard disk drives.

Another problem that is often encountered is missing backup copies. For instance, backups are only stored on a hard disk drive in one location - which might lead to data loss, e.g. in case of a fire.

Safe and reliable solutions are non-trivial to set up and require a high level of knowledge to operate.

Secondly, most of these solutions create backups directly to a writeable storage medium from the client computer and do not prevent data loss in case of ransomware infections~\cite{young-cryptovirology}.

\subsection{Solution}

The problems listed in the previous sections should be solved by a fast, easy to use, and secure backup system. Such a system should combine privacy benefits of local backups with automated data distribution, to provide high safety guarantees.

The target users of such a system are individuals (e.g. families) and small enterprises.

\section{Goals and Tasks}
% TODO: Rephrase, this paragraph was copied from https://eprints.hsr.ch/523/1/Data%20over%20DAB.pdf !
This section presents the revised goals and provides a rationale for the deviations from the original goals.

\subsection{Initial Goals}
The initial goals of this study project where specified by us in cooperation with Prof.~Farhad~Mehta in the \nameref{sec:task-description}.

\subsection{Revised Goals}
This section lists the revised goals that were specified during the beginning of the project. All deviations from the \nameref{sec:task-description} are noted in the following Chapter \fullref{sec:deviations-from-original-goals}.
% leicht angepasst, beschreibung im nächsten abschnitt
% sehr früh im projekt angepasst, sobald abgezeichnet

\begin{enumerate}
    \item Elaboration of following issues in a theoretical concept and architecture:
        \begin{enumerate}
            \item joining of nodes
            \item planned and unplanned leaving of nodes
            \item distribution of data within the network
            \item uploading data into the distributed system
            \item addressing within the distributed system
            \item retrieving stored data
            \item scalability for up to several 100 nodes where every node can store a data volume of up to 2 terabytes.
        \end{enumerate}
    \item Evaluation of an appropriate implementation language for the prototype
    \item Implementation of a prototype, demonstrating the core features as specified in the concept paper.
\end{enumerate}

\subsection{Deviations from the Original Goals}\label{sec:deviations-from-original-goals}

\subsubsection{Degree of Redundancy}
While researching data redundancy strategies, we realised that the full specification and implementation of \gls{client-m-replication} is not a feasible goal during the study project. The reason for this is its complexity and the limited time of the study project, as discussed in Chapter \fullref{sec:fundamental-design-decisions}.

\subsubsection{Simplifications for Prototype}
Due to time constrains of the study project, it was not possible to implement the full specified architecture in the prototype. Hence, we decided to implement the core backup, restore and distribution mechanisms, to prove that the concept works.

Simplifications for the prototype are discussed in Chapter~\fullref{sec:our-approach}.

\section{State of the Art}\label{sec:state-of-the-art}
% previous / related work in this area
In this section, we describe previous work and existing applications in this area.

\subsection{Backup Applications}
During our research, we primarily focused on software that was either described in academic papers or available under open source licenses.

\paragraph{Borg Backup} \cite{borg-backup} was the most promising candidate of a deduplicating, encrypting backup system, providing most of the described features.

Borg can also create backups to remote locations (e.g. over SSH) where a server-mode Borg instance is running on the remote location. The downside of this implementation is that the \gls{client} still needs full write access to the backup server and hence would permit malware to delete backups possibly.

\paragraph{Rdedup} \cite{rdedup} is an implementation of a backup software similar to Borg in Rust. It is still in an early development stage and is not yet able to create backups to remote destinations.

\paragraph{Rsync,} \cite{rsync} designed initially for file synchronisation, is also often used to create backups today. By using filesystem-hardlinks, it has file-deduplicating capabilities and can also synchronise files to remote locations, so that old backup cannot be modified.

The two most significant downsides of rsync are that it must be combined with several other applications (e.g. ssh, bash scripts, ceph) to provide a secure backup solution and that it is relatively complicated to set up and configure.

\subsection{Peer-to-Peer Backup Storage}
There exist different approaches to create distributed, encrypted peer-to-peer storage systems. Two notable representatives that create such a system decentralised over the internet are Tahoe-LAFS \cite{tahoe-lafs} and IPFS \cite{ipfs}.

These systems distribute data over multiple (third) parties, consuming and providing data storage. As such, the data security is based on encryption algorithms only.

Another well-researched market is that of distributed filesystems, with prominent representatives as the Andrew File System~\cite{afs} and ceph~\cite{ceph}. These filesystems are commonly used in low latency, directly connected environments as data centres.

There also exist multiple concept papers on peer-to-peer distributed backup storages. The reports ''Adaptive Redundancy Management for Durable P2P
Backup''~\cite{p2p-redundancy} and ''On Scheduling and Redundancy for P2P Backup''~\cite{p2p-scheduling} examine ways how such a system might distribute data in an efficient manner.

\subsection{Goals}
% Discuss Main goal of the paper
% Details are covered in Chapter "Our Approach"
% Summarize the Goals as described in the Task description and clarify it.
% Technology?
In our paper we focus mainly on a concrete backup software, introducing a practical solution that might be used by small enterprises and individuals. The architecture should provide developers with a clear guideline on how to implement such a system. Additionally, the prototype demonstrates a minimal implementation of such a system.
