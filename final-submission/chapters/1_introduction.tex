\chapter{Introduction}
\label{sec:introduction}

\section{Motivation}
% Rough introduction of the topic. Details are covered in Chapter "Background"

% TODO: Rephrase, this paragraph was copied from https://eprints.hsr.ch/523/1/Data%20over%20DAB.pdf !
With this chapter we legitimate this work, explain why this work is valuable and how the products of this labour can be brought to use for good.

\subsection{Present situation}
% What (free) backup solutions exist and why do we need another one?
% Has someone else published the same idea before?

\subsubsection{Problem}
% What problem do we solve with our proposal? / Why do we write this concept? New idea?
% Which problems have we found / are common for this topic?

\subsection{Possible applications}

\section{Goals and Tasks}
% TODO: Rephrase, this paragraph was copied from https://eprints.hsr.ch/523/1/Data%20over%20DAB.pdf !
This section presents the revised goals and provides a rationale for the deviations from the original goals.

\subsection{Initial Goals}
The initial goals of this study project where specified by us in cooperation with Prof.~Farhad~Mehta in the \nameref{sec:task-description}.

\subsection{Revised Goals}
This section lists the revised goals that were specified during the beginning of the project. All deviations from the \nameref{sec:task-description} are noted in the following Chapter \fullref{sec:deviations-from-original-goals}.
% leicht angepasst, beschreibung im nächsten abschnitt
% sehr früh im projekt angepasst, sobald abgezeichnet

\begin{enumerate}
    \item Elaboration of following issues in a theoretical concept and architecture:
        \begin{enumerate}
            \item joining of nodes
            \item planned and unplanned leaving of nodes
            \item distribution of data within the network
            \item uploading data into the distributed system
            \item addressing within the distributed system
            \item retrieving stored data
            \item scalability for up to several 100 nodes where every node can store a data volume of up to 2 terabytes.
        \end{enumerate}
    \item Evaluation of an appropriate implementation language for the prototype
    \item Implementation of a prototype, demonstrating the core features as specified in the concept paper.
\end{enumerate}

\subsection{Deviations from the Original Goals}\label{sec:deviations-from-original-goals}

\subsubsection{Degree of Redundancy}
While researching data redundancy strategies, we realised that the full specification of \gls{client-m-replication} is not a feasible goal during the study project. The reason for this is its complexity and the limited time of the study project, as discussed in Chapter \fullref{sec:fundamental-design-decisions}.

\subsubsection{Simplifications for Prototype}
Due to time constrains of the study project, it was not possible to implement the full specified architecture in the prototype. Hence, we decided to implement the core backup, restore and distribution mechanisms, to prove that the concept works.

Simplifications for the prototype are discussed in Chapter~\fullref{sec:our-approach}.

\section{State of the Art}
% previous / related work in this area

\subsection{Goals}
% Discuss Main goal of the paper
% Details are covered in Chapter "Our Approach"
% Summarize the Goals as described in the Task description and clarify it.
% Technology?

\subsubsection{Architecture}

\subsubsection{Prototype}



\subsection{Satisfaction}
% What goals were satisfied? How and why?
% Details are covered in Chapter "Discussion and Conclusion"
