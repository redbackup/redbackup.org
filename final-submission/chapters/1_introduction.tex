\chapter{Introduction}
\label{sec:introduction}

\section{Motivation}
% Rough introduction of the topic. Details are covered in Chapter "Background"

% TODO: Rephrase, this paragraph was copied from https://eprints.hsr.ch/523/1/Data%20over%20DAB.pdf !
With this chapter we legitimate this work, explain why this work is valuable and how the products of this labour can be brought to use for good.

\subsection{Present situation}
% What (free) backup solutions exist and why do we need another one?
% Has someone else published the same idea before?

\subsubsection{Problem}
% What problem do we solve with our proposal? / Why do we write this concept? New idea?
% Which problems have we found / are common for this topic?

\subsection{Possible applications}

\section{Goals and Tasks}
% TODO: Rephrase, this paragraph was copied from https://eprints.hsr.ch/523/1/Data%20over%20DAB.pdf !
In the following sections, the revised goals and reasons for deviations of the original goals are presented. The context of these goals is apparent from the \nameref{sec:task-description}.

\subsection{Initial Goals}

The initial goals of this thesis where specified by the authors in cooperation with Prof.~Farhad~Mehta and are also defined in the \nameref{sec:task-description}. 

\subsection{Revised Goals}
\begin{enumerate}
    \item Writing of a theoretical concept to address following issues:
        \begin{enumerate}
            \item joining of nodes
            \item planned and unplanned leaving of nodes
            \item distribution of data within the network
            \item uploading data into the distributed system
            \item addressing within the distributed system
            \item retrieving stored data
            \item scalability for up to several 100 nodes where every node can store a data volume of up to 2 terabytes.
        \end{enumerate}
    \item Evaluation to find an appropriate implementation language for the prototype
    \item Implementation of a prototype, demonstrating the core features as specified in the concept paper.
\end{enumerate}

\subsection{Deviations from Original Goals}

\subsubsection{Details of Defined Degree Redundancy Distribution}

While researching redundancy strategies for the prototype, we realised that the full specification of a defined-degree redundancy was not a feasible goal during the study project, because of the high complexity and limited time.

We therefore did not specify any details of this mechanism, but designed the architecture such that it would be extensible in the future.

\subsubsection{Simplifications for Prototype}

Due to time constrains of the study project, it was not possible to implement the full specified architecture in the prototype. Hence, we decided to implement the basic backup, restore and distribution mechanisms.

More details on prototype simplifications can be found in Chapter~\ref{sec:our-approach}.

\section{State of the Art}
% previous / related work in this area

\subsection{Goals}
% Discuss Main goal of the paper
% Details are covered in Chapter "Our Approach"
% Summarize the Goals as described in the Task description and clarify it.
% Technology?

\subsubsection{Architecture}

\subsubsection{Prototype}



\subsection{Satisfaction}
% What goals were satisfied? How and why?
% Details are covered in Chapter "Discussion and Conclusion"
