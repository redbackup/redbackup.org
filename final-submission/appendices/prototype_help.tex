\lstset{
  basicstyle=\ttfamily,
  columns=flexible,
  frame=single,
  breaklines=true,
  postbreak=\mbox{\textcolor{red}{$\hookrightarrow$}\space},
}

\section{Prototype Command Line Interface}\label{sec:prototype-command-line-interface}
\subsection{Client}
\begin{lstlisting}
$ redbackup-client-cli --help
redbackup client-cli 0.1.0
Raphael Zimmermann <dev@raphael.li>:Fabian Hauser <fabian@fh2.ch>
redbackup client

USAGE:
    redbackup-client-cli [OPTIONS] [SUBCOMMAND]

FLAGS:
        --help       Prints help information
    -V, --version    Prints version information

OPTIONS:
        --chunk-index-storage <chunk-index-storage>    Folder where chunk indices are stored. [default: /tmp/]
    -h, --node-hostname <node-hostname>                hostname of the node to contact [default: 0.0.0.0]
    -p, --node-port <node-port>                        port of the node to contact [default: 8080]

SUBCOMMANDS:
    create     Create a new backup
    help       Prints this message or the help of the given subcommands
    list       List available backups on the node.
    restore    List available backups on the node.
\end{lstlisting}

\subsubsection{Create}

\begin{lstlisting}
$ redbackup-client-cli create --help
redbackup-client-cli-create 
Create a new backup

USAGE:
    redbackup-client-cli create [OPTIONS] <expiration-date> <local-backup-dir>

FLAGS:
    -h, --help       
            Prints help information

    -V, --version    
            Prints version information


OPTIONS:
        --exclude-from <FILE>    
            Exclude multiple glob patterns from FILE. Define one pattern per line. Patterns are relative to the backup
            root, e.g. 'pictures/**/*.jpg'. For allowed glob syntax, see
            https://docs.rs/glob/0/glob/struct.Pattern.html#main

ARGS:
    <expiration-date>     
            the expiration date of this snapshot(format: %Y-%m-%dT%H:%M)

    <local-backup-dir>    
            Directories, that should be backuped
\end{lstlisting}

\subsubsection{List}
\begin{lstlisting}
$ redbackup-client-cli list --help
redbackup-client-cli-list 
List available backups on the node.

USAGE:
    redbackup-client-cli list

FLAGS:
    -h, --help       Prints help information
    -V, --version    Prints version information
\end{lstlisting}

\subsubsection{Restore}
\begin{lstlisting}
$ redbackup-client-cli restore --help
redbackup-client-cli-restore 
List available backups on the node.

USAGE:
    redbackup-client-cli restore <backup-id> <local-restore-dir>

FLAGS:
    -h, --help       Prints help information
    -V, --version    Prints version information

ARGS:
    <backup-id>            ID of the backup that should be restored
    <local-restore-dir>    Destionation, where the files should be restored to.
\end{lstlisting}

\subsection{Node}
\begin{lstlisting}
$ redbackup-node-cli --help
redbackup node-cli 0.1.0
Raphael Zimmermann <dev@raphael.li>:Fabian Hauser <fabian@fh2.ch>
redbackup node server

USAGE:
    redbackup-node-cli [OPTIONS] [--] [ARGS]

FLAGS:
    -h, --help       Prints help information
    -V, --version    Prints version information

OPTIONS:
    -k, --known-node <known-node>...    ip address and port (<ip-address>:<port>) of other known nodes in the network

ARGS:
    <ip>             IP to bind [default: 0.0.0.0]
    <port>           IP to bind [default: 8080]
    <storage-dir>    path to the storage directory [default: ./data/]
    <db-file>        path to the database file [default: db.sqlite3]
\end{lstlisting}