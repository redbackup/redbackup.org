% !TeX spellcheck = en_GB
\section{Fundamental Design Decisions} % TODO: better name?

We used the morphological box technique to explore different possible solutions.

The chosen option should be as simple as possible for the prototype developed in the study project but extensible for further adaption.

The following paragraphs reason the selected entry in each dimension.

\paragraph{Redundancy} 
We originally planned to support \emph{client m-replication}, which means that the client defines a custom degree of redundancy from 1 to the number of nodes. This, however, is a complex mechanism that requires sophisticated algorithms to work correctly and efficiently. For the prototype, we chose the more straightforward to implement option system \emph{n-replication}, where the degree of redundancy for the whole system is equal to the number of nodes in it. Changing this option in the future is hard because it requires fundamental changes in the replication process and the communication protocols.

\paragraph{Storage unit}
The idea of \emph{chunks} come from Borg Backup. Files are partitioned into chunks using a rolling hash which allows deduplication as well as space efficient backups for large files \cite{borg-data-structures}. These are desired properties in a backup system to minimise network and disk usage.

\emph{Encrypting chunks} means that deduplication of the same file coming from different users is not possible anymore but is a necessity for privacy. Encryption is not trivial and requires a user concept that is out of scope of the prototype developed in the study project.
We chose the \emph{plain filesystem} option for the study project to simplify the client implementation. Supporting \emph{encrypted chunks} in the future is possible by just modifying the client.

\paragraph{Role of the management}
The \emph{one in charge} option is the most straightforward option to implement but conflicts with many intentions of the administrator (See \fullref{sec:adminstrator-intention}). We also intended to avoid a single point of failure. We chose the option \emph{autonomous replication} because it guarantees that replication is always ensured and keeps communication relatively simple.

\paragraph{Storage backend}
Using the file system is the simplest possible solution for the study project and therefore selected option. Adding support for other backends in the future is still possible since the storage component is an isolated part of the architecture (See \fullref{sec:component-storage})

\paragraph{Removal of old backups}
We propose to use a fixed \emph{physical time} that must be specified on backup creation. After expiration, the backup data may be removed by a garbage collector. This may be extended to allow only mutual garbage removal in the future.

A significant problem that \emph{physical time} addresses is the safety of backup data in case a user computer is infected with malware. An illicit application might command the removal of backups, or create new backups to initiate a garbage collection process to free storage capacity.

Nevertheless, the use of \emph{physical time} has the downside of possible data loss due to wrong system times. To lessen this risk, the system should use multiple distinct upstream time-servers. This is given with a high probability, as the proposed redundancy model motivates users to expand the system across multiple physical locations. Furthermore, the client, nodes and management should verify a reasonable accurate time when communicating mutually.

\paragraph{Programming language / ecosystem}
A complete language evaluation can be found in \fullref{sec:language-evaluation}.


\begin{sidewaystable}
	\centering
	\caption[Morphological Box]{Morphological Box}
	\label{tbl:morphological-box}
    \begin{tabu}{X | X X X X}
		\hline
          \textbf{Redundancy}
          & No redundancy
          & Client m-replication: The client defines a custom degree of redundancy (from 1 to the number of nodes).
          & System m-replication: The administrator defines the degree of redundancy for the whole system (from 1 to the number of nodes).
          & \textbf{System n-replication}: The degree of redundancy for the whole system is equal to the amount of nodes in the system.
          \\ \hline

          \textbf{Storage unit}
          & \textbf{Plain files}
          & Encrypted files
          & Chunks: Cut files into multiple parts and store these individually.
          & Encrypted chunks: Same as chunks, but every chunk is individually encrypted.
          \\ \hline


          \textbf{Role of the management}
          & One in charge: The management knows and controls everything (e.g. the location of every file/chunk).
          & Configuration only: The management must be up for administrative tasks only. The nodes are mostly autonomous.
          & \textbf{Autonomous replication}: The management must be available for most of the tasks but replication also works if the management is down.
          & No management: Every node is completely autonomous.
          \\ \hline


          \textbf{Storage backend}
          & \textbf{Plain filesystem}: Just store all files/chunks as files in one directory with a unique identifier.
          & Database: Use an existing database solution (e.g. Git, Redis, RocksDB).
          & Cloud Storage: A proxy to a cloud storage provider (e.g. Amazon S3).
          & Custom: An optimized version of the plain file system option with optimised indexing and compression.
          \\ \hline


          \textbf{Removal of old backups}
          & \textbf{Physical time}: Data is removed on a specified physical time.
          & User command: The user commands removal of data.
          & Free storage: Data is removed, as soon as capacity issues occur.
          & Physical time with mutual agreement: All nodes must agree before data is removed.
          \\ \hline


          \textbf{Programming language / ecosystem}
          & \textbf{Rust}
          & Go
          & Erlang
          & 
          \\ \hline
	\end{tabu}
\end{sidewaystable}

\subsection{Hash Collisions}
To achieve deduplication and space-efficient backups for large files, as discussed above, a file/chunk identifier must be derived from the actual file/chunk contents. 
A common mechanism used to derive identifiers from binary data are cryptographic hash functions. Most cryptographic hash functions produce a message digest having a fixed size (e.g. SHA-256\cite{sha-256} produces a 256-bit digest) for a message with an arbitrary length, which can theoretically lead to collisions.
Perfect hash functions do not have this property because their input message size is fixed and equal to the size of the resulting message digest. A perfect hash function is not practical in our case due to the large message digests.
With cryptographic hash functions, collisions are possible but unlikely. Assuming that the applied function does produce equally distributed results, the probability can be calculated based on the birthday problem\cite{birthday-attack} as follows, where $p$ is the number of chunks in the system and $n$ the size of the message digests:

\[
P(p, n) = \frac{p^2}{2^{n+1}}
\]

Assuming we have $p=30^{21}$ files/chunks the system (which is equivalent to two billion years of music assuming each chunk has a size of one byte\cite{seagate-zetabyte}) and using the SHA-256 algorithm, the probability of collisions is about $4.72 \cdot 10^{-16}$, which is highly improbable and may therefore be neglected.

If, however, a collision would happen after all, for example, if the used cryptographic hash function is flawed or the unlikely event occurs, it results in data loss.

In theory, we could detect collisions on the client. To do so, every time an identifier is calculated, the client must verify that if a file with the same identifier exists already in the system, it has the exact same contents. If the contents differ, its a collision.This approach requires a lot of network traffic and can slow down the backup process significantly.

Another place to detect collisions is on the node. A node can verify if the contents of a given file/chunk are equal to the contents already present in the system. The downside of this approach is that it requires the client always to send the full contents of every file, which means there is a lot of additional network traffic.

Both of the described approaches for collision detection have significant costs that are not practical.

As for the study project, we use the SHA-256 algorithm\cite{sha-256} and neglect the risk of hash collisions due to its low probability. Nonetheless, we prepare all protocols and components to use an interchangeable mechanism for the calculation and transmission of file/chunk identifiers.
