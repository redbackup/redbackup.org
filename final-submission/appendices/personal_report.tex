% !TeX spellcheck = en_GB
\section{Personal Reports}

\subsection{Raphael Zimmermann}

The project was a fascinating intellectual challenge, which allowed me to use lots of my experience and knowledge acquired while studying.

Finding an appropriate way of documenting a complex software system was one of the significant challenges for me. It was interesting to attend the lecture "Application Architecture" in parallel which focuses on this problem entirely. Unfortunately, many interesting techniques were discussed relatively late in this course by which we already finished our specification. Reading about fault tolerance and networking patterns in the lecture "Advanced Patterns and Frameworks" also strengthened my confidence in our proposed architecture because we used many of them implicitly.

I played well together with Fabian even though we worked a lot apart. We found the right balance by discussing relevant tasks in person and doing more routine and detailed work separate and more focused.

The explorative nature of the project made it hard to come up with reasonable estimates. It was the right decision to keep sprints very short to stay agile. I also learned to appreciate the value of checklists, e.g. for sprint planning and completion.

\subsection{Fabian Hauser}

\enquote{It doesn't matter how beautiful your theory is...

If it doesn't agree with experiment, it's wrong.} - Richard Feynman
\vspace{1em}

This quote by Richard Feynman summarises pretty well, what made this project very interesting to me - building both, a high level architecture and a concrete, working prototype. During our studies, we learned a lot about tools to create and design systems, but never had the actual chance to build a larger project with them. I am very delighted for this opportunity!

One thing I especially enjoyed was working together with Raphael. As we come from different specialised backgrounds, with Raphael coming from a software and myself more from a system engineering side, our discussions regarding the application architecture were both very valuable and enriching.

Wanting to learn Rust for a while, this project seemed like the ideal opportunity to do so. For me, this proved to be one of the main challenges during this project thought, as Rust has a very steep learning curve. This mainly had impact on the programming productivity, although this probably would have been even worse with the choice of a functional language. Still, I am glad to have had the opportunity to learn Rust.