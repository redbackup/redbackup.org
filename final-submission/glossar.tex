\newglossaryentry{administrator}
{
    name={administrator},
    description={A person that is in charge of managing the system (e.g. add / remove nodes).}
}
\newglossaryentry{user}
{
    name={user},
    description={A registered user with valid credentials who wants to backup and restore its data. Has a client installed on his local computer.}
}
\newglossaryentry{node}
{
    name={node},
    description={A core participant in the system that manages \glspl{chunk} (See \fullref{sec:component-node}).}
}
\newglossaryentry{designated-node}
{
    name={designated node},
    description={A \gls{node} that is selected by a \gls{client} to send backup data to or restore data from.}
}
\newglossaryentry{new-node}
{
    name={new node},
    description={A \gls{node} scheduled to be integrated into the system, that was not part of that or any other system before.}
}
\newglossaryentry{leaving-node}
{
    name={leaving node},
    description={A \gls{node} that is scheduled to leave the system permanently.}
}
\newglossaryentry{sending-node}
{
    name={sending node},
    description={A \gls{node} that is sending data to a client or another node.}
}
\newglossaryentry{node-identifier}
{
    name={node identifier},
    description={A unique identifier assigned to any node to distinguish it from other \glspl{node}.}
}
\newglossaryentry{location}
{
    name={location},
    description={Multiple \glspl{node} can be associated with a (physical) location to ensure redundancy is met over multiple physical locations.}
}
\newglossaryentry{node-metadata}
{
    name={node metadata},
    description={A set of data that describes a \gls{node} (e.g. its state, storage capacity). The set of data differs depending on the context.}
}
\newglossaryentry{storage}
{
    name={storage},
    description={Component in charge of persisting data (see \fullref{sec:component-storage}).}
}
\newglossaryentry{serialised-chunk-index}
{
    name={serialised chunk index},
    description={A serialised version of the \gls{chunk-index}, that might be split into multiple chunks, created on backup and used for restore.}
}
\newglossaryentry{chunk-content}
{
    name={chunk content},
    description={The binary data a chunk represents.}
}
\newglossaryentry{chunk}
{
    name={chunk},
    description={A piece of data in the system, consisting of binary data (\gls{chunk-content}), a unique identifier (\gls{chunk-identifier}) and an \gls{expiration-date}.}
}
\newglossaryentry{root-handle}
{
    name={root handle},
    description={A flag to mark (the first part of) a \gls{serialised-chunk-index} to find all chunks associated with a backup for a restore.}
}

\newglossaryentry{chunk-identifier}
{
    name={chunk identifier},
    description={A unique identifier of a \gls{chunk} that is derived from the corresponding \gls{chunk-content}, e.g. using hash functions.}
}
\newglossaryentry{chunk-index}
{
    name={chunk index},
    description={A \gls{client}-side data structure that stores file attributes, folders and the file to \glspl{chunk} mapping.}
}
\newglossaryentry{chunk-table}
{
    name={chunk table},
    description={A data structure on a \gls{node} containing all \gls{chunk-identifier} and \glspl{expiration-date} of all \glspl{chunk} managed by this \gls{node}.}
}
\newglossaryentry{file}
{
    name={file},
    description={A document on a client system that shall be backed up. A file is represented in the system by one or more \glspl{chunk}.}
}
\newglossaryentry{expiration-date}
{
    name={expiration date},
    description={Date, until which a given \gls{chunk} must be kept in the system.}
}

\newglossaryentry{client}
{
    name={client},
    description={A piece of software that runs on a users computer, in charge of creating and restoring backups (see \fullref{sec:component-client}).}
}

\newglossaryentry{management}
{
    name={management},
    description={A component that orchestrates the configuration of the system (see \fullref{sec:component-management}).}
}

\newglossaryentry{system}
{
    name={system},
    description={The whole redbackup system as described in \fullref{sec:specification}.}
}

\newglossaryentry{node-cache}
{
    name={node-cache},
    description={A list of contact information of nodes in the system that each client buffers in case the management is unavailable.}
}

\newglossaryentry{node-state}
{
    name={node state},
    description={The intrinsic or extrinsic state of any node (see \fullref{sec:component-node}).}
}

\newglossaryentry{medium}
{
    name={medium},
    description={Particular form of storage for files, e.g. hard disks or magnetic tape.}
}

\newglossaryentry{message}
{
    name={message},
    description={A unit of communication between two parties (see \fullref{sec:messages}).}
}

\newglossaryentry{message-payload}
{
    name={message payload},
    description={Actual data of a \gls{message} (see \fullref{sec:messages}).}
}
\newglossaryentry{message-header}
{
    name={message header},
    description={Meta information of a \gls{message} that define operating parameters, consisting of multiple \glspl{header-field}.}
}
\newglossaryentry{header-field}
{
    name={header field},
    description={One piece of meta information that is part of a \gls{message-header}, consisting of a key and value.}
}
\newglossaryentry{client-m-replication}
{
    name={client m-replication},
    description={The \gls{client} defines a custom degree of redundancy from 1 to the number of \glspl{node} in the system. See \fullref{sec:fundamental-design-decisions}.}
}
\newglossaryentry{system-m-replication}
{
    name={system m-replication},
    description={\Glspl{node} replicate \glspl{chunk} to m other \glspl{node}. The degree of redundancy is defined globally by the administrator. See \fullref{sec:fundamental-design-decisions}.}
}
\newglossaryentry{system-n-replication}
{
    name={system n-replication},
    description={\Glspl{node} replicate \glspl{chunk} to all other \glspl{node}. As a result, the degree of redundancy is equal to the number of \glspl{node} in the system. See \fullref{sec:fundamental-design-decisions}.}
}